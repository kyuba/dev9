\documentclass[a4paper,twoside,titlepage]{article}
\usepackage[top=3cm,bottom=3cm,left=3cm,right=3cm]{geometry}
\usepackage{url}

\renewcommand\familydefault{\sfdefault}

\title{Dev9}
\author{Magnus Achim Deininger}

\begin{document}
\maketitle

\section*{Preface}
dev9 is a /dev manager for Linux systems. The idea is to implement a 9p2000.u
filesystem in userspace to do the tasks of the old `devfs' filesystem.

\section*{Introduction}
Currently, most Linux systems have a number of options for managing the special
/dev directory. Typical desktop distributions, for example, tend to use the
`udev' daemon, whereas embedded devices often have a static /dev, or use `mdev'.
Some desktop distributions may also use udev after the system has booted up, but
rely on mdev while the system is in an autoconfiguring initramfs. Older or
modern but more exotic Linux distributions on the other hand may use the special
`devfs' pseudo-filesystem.

There's a number of issues with these management methods: udev is theoretically
nice, but it's easily integrated poorly into the system startup sequence, thus
resulting in poor bootup speeds when used with excessive amounts of shell
scripting for coldplugging of devices. Static /dev management is very swift, but
it must be kept up-to-date manually or by periodically running maintenance
tasks, and it also has the disadvantage of not allowing for nice features like a
more meaningful device naming scheme. Devfs was up to a nice start, but it
eventually suffered from bitrot and has been removed from most kernel patchsets.
Mdev is slick and fast, but it isn't supposed to be kept running to update /dev
if new devices are added to the system while it is running, which is an issue
that a static /dev suffers from as well.

To aleviate these issues, we're trying to build a programme that is sort of a
combination of devfs and udev; that is, a device-managing filesystem implemented
in userspace, using the 9p2000.u protocol. This has a number of advantages,
which will be pointed out in this paper.

\section{Concept}
This chapter will sum up the general idea of how dev9 is going to work, and how
it's supposed to be used by init programmes and users.

\subsection{Operation}
Dev9 itself will be a single daemon, although it may be split into multiple
binaries, depending on whether this provides an advantage with respect to memory
consumption. Upon initialisation, it will read its configuration files -- on
disk first, then things off the command-line, to allow for init-specific
overrides. It will also mount /proc and /sys, as well as its own 9p2000.u
filesystem under /dev, possibly also loading required kernel modules so these
operations succeed and forking afterwards, so that the calling programme can
resume. This operation should be very swift to perform.

The /dev filesystem will, by default, be mounted by the kernel by directly
passing a set of fds, so that no write access is required anywhere to create an
appropriate socket.

After that, it will open a netlink connection to the kernel, to listen for
uevents, and search through /sys to make the kernel emit uevents for hardware
that is already present. The programme will listen for uevents, and create
internal data structures to hold the devicefiles in the /dev filesytem as
needed. While the programme is scanning /sys, it will already answer all queries
for present nodes in /dev, but it will defer answers to nodes that haven't been
created yet until the scan is complete.

This method of operation has a number of advantages: It's fairly simple to do,
algorithmically, and it allows for an `instant-on' /dev filesystem without any
coordination overhead from other parts of the system, since all the nodes appear
to be always present right away, at least if the nodes or the directory they
reside in are known in advance. It also removes the need for any symlinking
between device nodes, as is used by udev and other management methods. This
should remove path resolution overhead, especially since there is little to be
read from devicefiles, other than their access bits and the major/minor numbers.

Another advantage is that being able to defer the replies to access requests for
device nodes under /dev allows us to implement some of the features devfs had,
for example it could be used to bring back the device-node based module
autoloading, which was a quite useful feature to have. Security-wise, it could
also be used to introduce special security checks when trying to access some
directories or add certain types of logging.

\subsection{Usage}
The daemon, or one of its subprogrammes will encapsulate all the mounting logic
required to mount itself, as well as /proc and /sys. This behaviour will be
toggleable from the command-line, as well as the configuration files.

The stdin/stdout file descriptors may be put into two modes: either they may be
used for logging and daemon control, with an S-expression based protocol for
reduced overhead, or they may be requested to host the 9p2000.u volume.

Init programmes and users likewise, will in most cases only need to call the
dev9 binary, and it should try to `just work'.

\subsection{Configuration}
All the configuration data will be stored as S-expressions in textual form,
because S-expressions are easy, and fast to parse. Unlike XML or other textual
storage methods, they're also typed and a parser is easily implemented in under
10k of object code.

Configuration data should be stored in /etc/dev9 or the appropriate location for
the file hierarchy system in use. Additionally, the daemon will need to accept
its configuration files via the command-line, so that an init or a user is able
to adjust the configuration as needed.

\section{OS requirements}
This project is, by its very definition, strictly Linux-only, so portability is
not going to be any issue whatsoever, save for portability between different
Linux kernel versions and architecture-specific issues.

\subsection{Kernel requirements}
Like any 9p2000.u filesystem, this project will require a running kernel with
support for the 9p2000.u (network) filesystem protocol. Since patches for this
have been included in the vanilla 2.6.x kernel sources for a number of versions
now, this shouldn't pose much of a problem.

Additionally, this programme will require support for sysfs and netlink in the
kernel. Both of these would also be required for udev to work.

\subsection{Library requirements}
To achieve minimal code size, dev9 will use the Curie and Duat libraries. This
should make for extremely small binaries, as well as provide the code needed to
deal with S-expressions.

\end{document}
